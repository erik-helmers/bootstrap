% Created 2023-03-11 sam. 19:44
% Intended LaTeX compiler: pdflatex
\documentclass[11pt]{article}
\usepackage[utf8]{inputenc}
\usepackage[T1]{fontenc}
\usepackage{graphicx}
\usepackage{longtable}
\usepackage{wrapfig}
\usepackage{rotating}
\usepackage[normalem]{ulem}
\usepackage{amsmath}
\usepackage{amssymb}
\usepackage{capt-of}
\usepackage{hyperref}
\usepackage{proof}
\usepackage{mathpartir}
\usepackage{amsmath,amssymb,amsthm,textcomp}
\usepackage{listofitems}
\usepackage{bssetup}
\usepackage{xparse}
\author{Erik Helmers}
\date{\today}
\title{Bootstrap}
\hypersetup{
 pdfauthor={Erik Helmers},
 pdftitle={Bootstrap},
 pdfkeywords={},
 pdfsubject={},
 pdfcreator={Emacs 28.2 (Org mode 9.5.5)}, 
 pdflang={English}}
\begin{document}

\maketitle
\tableofcontents



\section{Théorie}
\label{sec:org81b7bbd}

\subsection{Syntaxe}
\label{sec:orga7ec310}

\begin{decl}{\te, \tty, \tk}
 \drule{\te: \tty                 }{annotated term}
 \drule{\tv                       }{variable}
 \drule{\tlam{x}{\te}             }{lambda}
 \drule{\te\ \te[2]               }{application}
 \drule{\tpi{x}{\tty}{\tty[2]}    }{pi type}
 \drule{(\te, \te[2])             }{tuple}
 \drule{\tfst{\te}                }{fst}
 \drule{\tsnd{\te}                }{snd}
 \drule{\tsig{\tv}{\tty}{\tty[2]} }{sigma type}
 \drule{\tstar                    }{type of types}
\end{decl}


where \(\te\), \(\tty\), \(\tk\) represent general expressions, types and kinds respectively.

\subsection{Contexte}
\label{sec:org9f33234}

\begin{decl}{\ctx}
    \drule{\epsilon}{empty context}
    \drule{\ctx, \tv:\vty}{adding a variable}
\end{decl}

\begin{mathpar}
\inferrule*{  }{ \ctxValid{\epsilon} }
\inferrule*
    { \ctxValid{\ctx} \\ \ctx \ctxmap \vty \tycheck \vstar }
    { \ctxValid{\ctx, \tv : \vty} }
\end{mathpar}

\subsection{Evaluation}
\label{sec:org3691e2e}

\begin{decl}{\ve, \vty}
    \drule{\vn                         }{neutral term}
    \drule{\vlam {\vv} {\ve}           }{lambda}
    \drule{\vtup {\ve} {\ve[2]}        }{tuple}
    \drule{\vstar                      }{type of types}
    \drule{\vpi {\tv} {\vty} {\vty[2]} }{dependent function space}
    \drule{\vsig {\tv}{\vty}{\vty[2]}  }{dependent pair space }
\end{decl}

\begin{decl}{\vn}
    \drule{\vv                         }{variable }
    \drule{\napp{\vn}{\ve}             }{neutral app}
    \drule{\nfst{\vn}                  }{neutral first projection}
    \drule{\nsnd{\vn}                  }{neutral second projection}
\end{decl}

\begin{mathpar}


\inferrule*[right=(Star)] {\\} { \tstar \evalsto \vstar} \and

\inferrule*[right=(Var)]{\\}{ \tv \evalsto \vv} \and

\inferrule*[right=(Ann)]
  {\te \evalsto \ve}
  {\te: \tty \evalsto \ve}
\and

\inferrule*[right=(Lam)]
    {\te\evalsto \ve }
    { \tlam{\tv}{\te} \evalsto \vlam{x}{\ve}}
\and
\inferrule*[right=(Tuple)]
    {\te\evalsto \ve \\ \te[2] \evalsto \ve[2] }
    { (\te, \te[2]) \evalsto (\ve, \ve[2])}
\and

\inferrule*[right=(App)]
  {\te \evalsto \vlam{\vv}{\ve} \\ \ve[1][ x \mapsto \te[2]] \evalsto \ve[2]}
  { \te\ \te[2] \evalsto \ve[2]}
\and
\inferrule*[right=(NApp)]
  {\te \evalsto \vn \\ \te[2] \evalsto \ve[2]}
  {\te \ \te[2] \evalsto \napp{\vn}{\ve[2]}}
\and

\inferrule*[right=(Fst)]
  {\te  \evalsto (\ve ,\ve[2])}
  { \tfst{\te} \evalsto \ve}
\and

\inferrule*[right=(Snd)]
  {\te \evalsto (\ve,\ve[2])}
  { \tsnd{\te} \evalsto \ve[2]}
\and

\inferrule*[right=(NFst)]
  {\te  \evalsto \vn}
  { \tfst{\te} \evalsto \nfst{\vn}}
\and

\inferrule*[right=(NSnd)]
  {\te  \evalsto \vn}
  { \tsnd{\te} \evalsto \nsnd{\vn}}
\and

\inferrule*[right=(Pi)]
  {\tty \evalsto \vty \\ \tty[2] \evalsto \vty[2]}
  {\tpi{\tv}{\tty}{\tty[2]} \evalsto \vpi{\vv}{\vty}{\vty[2]}}
\and

\inferrule*[right=(Sigma)]
  {\tty \evalsto \vty \\ \tty[2] \evalsto \vty[2]}
  {\tsig{\tv}{\tty}{\tty[2]} \evalsto \vsig{\vv}{\vty}{\vty[2]}}
\and


\end{mathpar}

\subsection{Typing}
\label{sec:orgb9ec553}

In the following, \(\te\tysynth \vty\) is an expression whose type synthezises to \(\vty\) while \(e \tycheck \vty\) is checkable.

\begin{mathpar}
\centering

\inferrule*[right=(Chk)] { \ctx \vdash \te \tysynth \vty }{ \ctx \vdash \te \tycheck \vty }
\and

\inferrule*[right=(Ann)]
  {\ctx \vdash \tty \tycheck * \\ \tty \evalsto \vty \\
   \ctx \vdash\te\tycheck \vty}
  { \ctx \vdash (\te:\tty) \tysynth \vty }
\and


\inferrule*[right=(Star)]{ }{ \ctx \vdash \tstar \tycheck \vstar }
\and


\inferrule*[right=(Var)] { \ctx(\tv) = \vty }{ \ctx \vdash \tv \tysynth \vty }
\and

\inferrule*[right=(Lam)]
  { \ctx,\tv : \vty \vdash\te\tycheck \vty[2] }
  { \ctx \vdash  \vlam{\tv}{\te} \tycheck \vpi{\tv}{\vty}{} \vty[2]}
\and

\inferrule*[right=(Tuple)]
  { \ctx \vdash\te\tycheck \vty \\  \ctx \vdash \te[2] \tycheck \vty[2]}
  { \ctx \vdash  (\te,\te[2]) \tycheck \vsig {\tv}{\vty}{\vty[2]}}
\and

\inferrule*[right=(App)]
  { \ctx \vdash\te\tysynth  \vpi{x}{\vty}{\vty[2]}  \\  \ctx \vdash \te[2] \tycheck \vty \\ \vty[2][x \mapsto \te[2]] \evalsto \vty[3] }
  { \ctx \vdash \te\ \te[2] \tysynth \vty[3]}
\and

\inferrule*[right=(Fst)]
  { \ctx \vdash\te\tysynth \vsig{x}{\vty}{\vty[2]}}
  { \ctx \vdash \tfst{\te} \tysynth \vty}
\and

\inferrule*[right=(Snd)]
  { \ctx \vdash\te\tysynth \vsig{x}{\vty}{\vty[2]} \\ \vty[2][x \mapsto \text{fst}\ e] \evalsto \vty[3] }
  { \ctx \vdash \tsnd{\te} \tysynth \vty[3]}
\and


\inferrule*[right=(Pi)]
   { \ctx \vdash \tty \tycheck \vstar \\ \tty \evalsto \vty \\ \ctx,\tv:\vty \vdash \tty[2] \tycheck \vstar }
   { \ctx \vdash \tpi{\tv}{\tty}{\tty[2]} \tycheck \vstar }
\and

\inferrule*[right=(Sigma)]
   { \ctx \vdash \tty \tycheck \vstar \\ \tty \evalsto \vty \\ \ctx,\tv:\vty \vdash \tty[2] \tycheck \vstar }
   { \ctx \vdash \tsig{\tv}{\tty}{\tty[2]} \tycheck \vstar }
\and
\end{mathpar}

Une reformulation équivalentes des règles \(\textsc{(Pi)}\) et \(\textsc{(Sigma)}\), plus adaptée pour l'implémentation :

\begin{mathpar}
\inferrule*[right=(Pi)]
   { \ctx \vdash \tty \tycheck \vstar \\ \tty \evalsto \vty \\ \ctx \vdash \tty[2] \tycheck \vpi{\tv}{\vty}{\vstar} }
   { \ctx \vdash \vpi{\tv }{ \tty}{\tty[2]} \tycheck \vstar }
\and

\inferrule*[right=(Sigma)]
   { \ctx \vdash \tty \tycheck \vstar \\ \tty \evalsto \vty \\ \ctx \vdash \tty[2] \tycheck \vpi{\tv}{\vty}{\vstar} }
   { \ctx \vdash \vsig{\tv}{\tty}{\tty[2]} \tycheck \vstar }
\and
\end{mathpar}
\end{document}